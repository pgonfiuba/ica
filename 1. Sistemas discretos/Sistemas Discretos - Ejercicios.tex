\documentclass[a4paper,12pt]{article}
\usepackage[utf8]{inputenc}
\usepackage{graphicx}\graphicspath{{images/}}
\usepackage{overpic} % Paquete necesario para superponer imágenes
%\usepackage{subcaption}
\usepackage[caption=false]{subfig}
\usepackage{textcomp}
\usepackage{amsmath}
\usepackage{amssymb} % Necesario para \mathbb
\usepackage{bm}
\usepackage[makeroom]{cancel}
\usepackage{nccmath}
\usepackage[spanish,es-latin]{babel}


%opening
\title{Guía de Ejercicios - Sistemas de Tiempo Discreto}
\author{}

\begin{document}

\maketitle

\section{Discretización por ZOH}

\begin{enumerate}
	\item Dado el sistema
	      \begin{align*}
		      \frac{dx}{dt} & =-a\, x + b\,u \\
		      y             & = c\, x
	      \end{align*}
	      Si la entrada es constante sobre períodos de tiempo $T$, obtener el sistema muestreado y evaluar cómo se mueven los polos según se varía $T$


	\item Derivar los sistemas de tiempo discreto correspondientes a los siguientes de tiempo continuo. Considerar el uso de un ZOH

	      \begin{enumerate}
		      \item
		            \begin{align*}
			            \frac{dx}{dt} & = \begin{bmatrix}
				                              0 & 1 \\ 0 & 0
			                              \end{bmatrix} x + \begin{bmatrix} 0\\1 \end{bmatrix} u \\
			            y             & = \begin{bmatrix}1 & 0 \end{bmatrix}
		            \end{align*}
		            Comprobar que es un doble integrador
		      \item
		            \begin{align*}
			            \frac{dx}{dt} & = \begin{bmatrix}
				                              -1 & 0 \\ 1 & 0
			                              \end{bmatrix} x + \begin{bmatrix} 1\\0 \end{bmatrix} u \\
			            y             & = \begin{bmatrix}0 & 1 \end{bmatrix}
		            \end{align*}
		            Comprobar que corresponde a un motor

		      \item \begin{align*}
			            \frac{dx}{dt} & = \begin{bmatrix}
				                              0 & 1 \\ -1 & 0
			                              \end{bmatrix} x + \begin{bmatrix} 0\\1 \end{bmatrix} u \\
			            y             & = \begin{bmatrix}1 & 0 \end{bmatrix}
		            \end{align*}
		            ¿Qué clase de sistema es?
		      \item \[ \frac{d^2y}{dt^2} + 3 \frac{dy}{dt}+2y = \frac{du}{dt}+3u \]
		      \item \[ \frac{d^3y}{dt^3} = u \]
	      \end{enumerate}
	\item Se asume que las siguientes ecuaciones en diferencias describen sistemas en tiempo continuo muestreados con un circuito \textit{zero-order hold} y período de muestreo $h$.
	      Determinar, si es posible, los correspondientes sistemas en tiempo continuo.

	      \begin{enumerate}
		      \item
		            \[
			            y(kh) - 0.5\,y(kh-h) = 6\,u(kh-h)
		            \]

		      \item
		            \begin{align*}
			            x(kh+h) & =\begin{bmatrix}-0.5 & 1    \\0    & -0.3\end{bmatrix}x(kh)+\begin{bmatrix}0.5 \\0.7\end{bmatrix}u(kh) \\
			            y(kh)   & =\begin{bmatrix}1 & 1\end{bmatrix}x(kh)
		            \end{align*}

		      \item
		            \[
			            y(kh) + 0.5\,y(kh-h) = 6\,u(kh-h)
		            \]
	      \end{enumerate}
	      ¿Es posible reconstruir siempre el sistema de tiempo continuo?

	\item Considere el sistema continuo dado por
	      \[
		      G(s) \;=\; \frac{1}{s^{2}(s+2)(s+3)}.
	      \]
	      Suponga que la entrada es una suma de impulsos en los instantes de muestreo, es decir,
	      \[
		      u(t) \;=\; \sum_{k=-\infty}^{\infty} \delta\!\big(t-kh\big)\,u(kh),
	      \]
	      donde $\delta(\cdot)$ es el impulso de Dirac y $h>0$ es el período de muestreo.

	      Determine la representación en tiempo discreto del sistema.


	\item Determinar la función de transferencia por pulsos (pulse-transfer function) del siguiente sistema en tiempo discreto:

	      \begin{align*}
		      x(kh+h) & = \begin{bmatrix} 0.5 & -0.2\\0 & 0\end{bmatrix}x(kh)+\begin{bmatrix}2 \\ 1 \end{bmatrix}u(kh) \\
		      y(kh)   & =\begin{bmatrix}1 & 0\end{bmatrix}x(kh)
	      \end{align*}

	\item Si $\beta<\alpha$ el siguiente sistema se conoce como red de adelanto
	      \[ G(s)=\frac{s+\beta}{s+\alpha}\]

	      Para el siguiente sistema discreto
	      \[ H(z)=\frac{z+b}{z+a}\]
	      determinar bajo qué condiciones funciona como red de adelanto


	\item Sea el sistema
	      \[ G(s)=\frac{s+b}{s+a}\]
	      Obtener las condiciones para que el sistema muestreado con $T$ tenga inversa estable

\end{enumerate}

\section{Análisis de sistemas discretos}
\begin{enumerate}
	\item Dado el siguiente sistema de tiempo continuo
	      \[G(s) = \frac{1}{s^2+1.4s+1}\]
	      Obtener el sistema $H(z)$ muestreado con ZOH para $T=0.4$. Calcular los márgenes de ganancia para ambos. ¿Qué ocurre con el mismo al variar el tiempo de muestreo?

	\item Obtener el error estacionario al escalón y a la rampa del siguiente sistema:
	      \[y_k = H(q) u_k = \frac{q-0.5}{(q-0.8)(q-1)} \]

	\item Dado el sistema discreto
	      \[
		      x(k+1)=\begin{bmatrix}
			      0 & 1 & 2 \\ 0& 0 & 3 \\ 0&0&0
		      \end{bmatrix}x(k)+\begin{bmatrix}
			      0 \\ 1 \\ 0
		      \end{bmatrix}
		      u(k)
	      \]

	      \begin{enumerate}
		      \item[(a)] Determine una secuencia de control $\{u(0),u(1),\dots\}$ tal que el sistema pase desde el estado inicial \(x^{\top}(0)=\begin{bmatrix}1 & 1 & 1\end{bmatrix}\) hasta el origen en un número finito de pasos.
		      \item[(b)] ¿Cuál es el número mínimo de pasos necesario para resolver el problema planteado en (a)?
		      \item[(c)] Explique por qué no es posible encontrar una secuencia de señales de control tal que el estado \(x^{\top}(0)=\begin{bmatrix} 1 & 1 & 1\end{bmatrix}\)
		            sea alcanzado partiendo desde el origen.
	      \end{enumerate}

	\item Obtener el sistema muestreado de
	      \[ G(s)=\frac{s+1}{s^2+0.2s+1}\]

	      Determinar los valores de $T$ para los que el sistema tendrá oscilaciones ocultas
\end{enumerate}


\end{document}
